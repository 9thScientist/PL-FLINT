\chapter{Análise e especificação}

\section{Análise do problema}
Em 1972 estava Dennis Ritchie a terminar o que se veio a tornar uma das mais importantes linguagens de programação, a linguagem C. Devido à sua idade e filosofia como linguagem imperativa de baixo nível, \textit{strings} não são suportadas naturalmente ou, melhor dizendo, não são suficientemente flexíveis para tratar, como é muitas vezes o caso, linguagens naturais.

É muito comum querer ter uma \textit{string} com conteúdo variável, um simples cumprimento como \texttt{"Olá NOME"} em que \texttt{NOME} representa uma outra \textit{string} com o nome do recetor. Para este efeito é frequentemente usada a função \texttt{sprintf} da biblioteca \texttt{<string.h>} que permite escrever vários dados numa string com um formato específico. Esta solução funciona e de facto é eficaz, mas é tediosa quando se pretende fazer a um conjunto significativo de linhas. E, devido à natureza estática de \textit{arrays} em C, não é fácil garantir que não se excede o limite máximo da \textit{string}.

\section{Objetivos}
Seria ideal imaginarmos uma ferramenta que simplificasse todo este processo de modelação e que garantisse uma maneira eficaz de construir um numeroso conjunto de \textit{strings} formatadas com dados variáveis. Essa ferramenta teria de ser capaz de transformar um modelo, em código C compilável e eficiente. Teria que usar \emph{strings dinâmicas}, mantendo sempre atenção para não permitir \textit{leaks} de memória. Estamos portanto a falar de um \emph{processador de modelos}, como será chamado em Português, ou \textit{template engine} na língua anglo-saxónica, em par com uma \emph{linguagem de modelação}, ou \textit{template language}.

É com este objetivo que desenvolvemos uma \textit{linguagem de modelação} e o seu correspondente \textit{processador de modelos}, que recebe o modelo especificado nessa linguagem e o converte em código C válido.
